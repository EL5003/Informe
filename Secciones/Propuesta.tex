\section{Propuesta}
\label{sec:propuesta}

\subsection{Cotizaciones}
\label{sec:disenos}

En una primera instancia, varios proveedores de dispositivos ópticos
\emph{DWDM} fueron contactados para estudiar distintas propuestas con
equipos distintos. Contar con varias de estas cotizaciones habría
permitido comparar los costos de los equipos que irán instalados en
los nodos de la red y así habría permitido poder estimar de mejor
manera el costo de capital del proyecto.

Los proveedores de redes contactados fueron:
\begin{itemize}
\item Alcatel
\item Ericsson
\item Huawei
\item Nokia NSN
\item Siemens
\item Sonda (representantes de Cisco en Chile)
\end{itemize}

A la par de esta cotización, se desarrolló un modelo probabilístico
para la programación de los dispositivos \emph{ROADM} que asegura un
valor de mínimo de disponibilidad de la red; esto determina
principalmente el \emph{SLA} estimado de la red (mayores detalles se
exponen en la sección \ref{sec:SLA}). Este modelo probabilístico se
programará en la red con topología \emph{WSON} y controlada a través
de dispositivos \emph{WSS}, haciendo que el requerimiento en
\emph{SLA} sea en efecto una variable de entrada para el diseño.

% Para la definición de la red, fue necesario definir las rutas que
% conectan cada par de \textit{DataCenters}.  Para obtener un modelo
% robusto a cortes, es necesario que exista más de una ruta entre cada
% par de puntos o almenos cierta cantidad que garantize los
% requerimientos predefinidos.

% La forma en que se diseño esta red, utiliza ciertas características
% intrínsicas de la red de fibra oscura actual. Por ejemplo se ciñe a
% las tasas anuales de corte para la fibra utilizada (AEG-10
% \textbf{CORREGIR}), y también a las capacidades requeridas por cada
% protocolo GE y (balblalbal)


% El conjunto de rutas alternativas debe cumplir un criterio de umbral
% de seguridad, si no lo cumple se agregarán más rutas alternativas
% hasta que se logre dicho objetivo.

Hasta el día 23 de julio de 2014, no ha habido respuesta de los
proveedores de sistemas completos. Por ello y para tener un sistema
cotizado para la fecha de entrega, se cotizó utilizando solamente la
lista de precios de Ericsson, con la cual se contaba gracias a
contactos del docente del taller con dicha empresa.

\subsection{Propuesta final}
\label{sec:ppfinal}

Al final, la propuesta que se presenta en este informe consiste en una
red óptica \emph{DWDM} modular, compuesto por dispositivos, jumpers y
racks. Estos equipos se instalarán en los \emph{DC} tal y como se
detalla en la sección \ref{sec:ingdetalle}.

En total, se cotizaron los equipos y en la cantidad mostrada en la
tabla \ref{tab:propfinal}.

\begin{table}[H]
  \centering
  \begin{tabular}{| c | r |}
    \hline
    Equipo & Cantidad \\
    \hline
    WSS MUX & 32 \\
    EDFA & 52 \\
    MUX & 12 \\
    DEMUX & 12 \\
    ODF & 18 \\
    Racks & 12 \\
    Transpondedores 10GE & 184 \\
    Transpondedores 4FC & 122 \\
    Fibra (jumpers) & 132 \\
    \hline
  \end{tabular}
  \caption{Cantidad de equipos cotizados.}
  \label{tab:propfinal}
\end{table}

\subsubsection{CAPEX}
\label{sec:capex}

El CAPEX del proyecto considera todos los insumos necesarios en los que se requiere invertir para poner el proyecto en marcha. Dentro de los nombrados en la sección \ref{sec:planificacion} y \ref{sec:ingdetalle} los equipos más importantes corresponden a las tarjetas WSS y a los transpondedores los cuales abarcan cerca del 80\% del total de la inversión. Los demás suministros necesarios corresponden a equipos Mux/Demux, ODF, jumpers, aparatos EDFA, y otros.

Para calcular el precio total de la inversión se considero una propuesta obtenida del proveedor \newline Ericsson$^\circledR$, dentro de su línea de productos MHL3000, la que corresponde a equipos necesarios para montar una red DWDM convencional.

El monto total de la inversión asciende a 4.253.320 \$USD. El detalle del CAPEX se encuentra en la tabla \ref{tab:capex}.

\begin{table}[H]
  \centering
  \begin{tabular}{| l | c | c | r |}
    \hline
    \textbf{Ítem} & \textbf{Precio Unit. (USD)} & \textbf{Unidades} & \textbf{Total USD} \\
    \hline
    WSS (Mux) & 22.000 & 32 & 704.000\\
    \hline
    Mux  & 5.000 & 12 & 60.000 \\
    \hline
    Demux & 5.000 & 12 & 60.000 \\
    \hline
    Transpondedores  & 10.000 & 306 & 3.060.000 \\
    \hline
    Jumpers  & 10 & 132 & 1.320 \\
    \hline
    Kit para cables & 500 & 6 & 3.000 \\
    \hline
    Racks  & 2.500 & 12 & 30.000 \\
    \hline
    Pre-Amp & 7.500 & 26 & 195.000\\
    \hline
    Boosters & 7.500 & 26 & 195.000\\
    \hline
    ODF & 2.500 & 18 & 45.000\\
    \hline    
    Total & & & $\approx \$4.293.320$ USD \\
    \hline
  \end{tabular}
  \caption{Detalle CAPEX del proyecto}
  \label{tab:capex}
\end{table}

\subsubsection{OPEX}
\label{sec:opex}

Los gastos de operación de un proyecto de telecomunicaciones consisten en los siguientes puntos que se considerarán para este proyecto específico:
\begin{itemize}
\item Horas hombre para operadores de la red
\item Contratos de soporte
\item Consumo de fuentes de poder
\end{itemize}

Los puntos anteriores se pueden estimar a precio de mercado. Esto es
lo que se hizo en la tabla \ref{tab:opex}.

\begin{table}[H]
  \centering
  \begin{tabular}{| l | c | c | r |}
    \hline
    \textbf{Ítem} & \textbf{Precio x 1 (USD por unidad)} & \textbf{Unidades x año} & \textbf{Total USD} \\
    \hline
    Horas h. operadores & 30 & 720 & 21.600 \\
    \hline
    Contratos de soporte (u) & 500.000 & 1 & 500.000 \\
    \hline
    Consumo de potencia (KWh) & 1.10 & 26.280 \footnotemark & 28.908 \\
    \hline
    Total & & & $\approx \$550.000$ USD \\
    \hline
  \end{tabular}
  \caption{Detalle OPEX de un año tipo (válido para el primer año)}
  \label{tab:opex}
\end{table}

\footnotetext{Este
  valor corresponde a un consumo promedio estimado de 500 W por todos
  los equipos en un datacenter, por horas en un año.}

El valor calculado de \emph{OPEX} tiene proporciones de
aproximadamente 4\%, 91\% y 5\% entre sueldo de operadores, contratos
de soporte y gasto energético, respectivamente.  
