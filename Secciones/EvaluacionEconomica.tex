\section{Evaluaci\'on Econ\'omica}\label{sec:evaluacion}

\subsection{CAPEX}

Para este punto de la planificación del proyecto, es importante
visualizar todos los componentes del proyecto y la cantidad de horas
hombre involucradas en la ingeniería e instalación de equipos. Es 
especialmente importante visualizar los componentes con los que no
se cuenta en un principio, siendo necesaria la ...

La instalación inicial de fibra tipo G.652.D permite ahorrarse este componente de la ...

% Monto total de las inversiones requeridas para la ejecución completa de un proyecto. Se 
% incluyen por lo tanto los siguientes ítems: 

% Equipos de telecomunicaciones, generalmente importados 
% Equipos de poder y clima (rectificadores, bancos de baterías, equipos de aire 
% acondicionado) 
% Materiales de instalación 
% Stock de unidades de repuesto 
% Licencias de software requeridas para la gestión de los equipos 
% Pre-ingeniería 
% Ingeniería de detalle 
% Instalación y Puesta en marcha 
% Administración del proyecto 
% Capacitación 
 
% Los primeros cuatro elementos se explican por sí solos. Algunos comentarios sobre los 
% restantes: 
 
% La pre-ingeniería considera las actividades previas a la ejecución del proyecto mismo como 
% visitas a terreno para definir ubicación física de los equipos, determinación de la existencia de 
% recursos de poder (alimentación continua -48 VDC) disponibles para los equipos, asignación de 
% automáticos, posiciones de las fibras ópticas en los ODF, etc. 
 
% En algunos casos, con proyectos que incluyan enlaces de microondas, la pre-ingeniería es 
% más compleja, ya que requiere definición de ubicación de estaciones de radio, pruebas de 
% despeje radioeléctrico, predicción de comportamiento de los radioenlaces, etc. 

