\section{Conclusiones}\label{sec:conclusiones}

Las redes ópticas pueden ser diseñadas de manera flexible usando
tecnologías que permiten multiplexar y manejar una gran cantidad de
señales de forma confiable y eficiente en una sola fibra óptica. Esto
se puede hacer de forma escalonada, instalando equipos que conformen
una red óptica \emph{DWDM}.

Los costos de estas implementaciones disminuyen día a día. Los
sistemas de fibra oscura han perdido terreno en el mercado por su baja
escalabilidad y poca flexibilidad. Desde un punto de vista técnico,
estas redes despilfarran muchos recursos en favor de su simplicidad de
instalación y menores costos.

Se han agregado funcionalidades al tendido inicial de fibra oscura que
se traducen en una mejor calidad de servicio, al idear una forma de
diseñar la red óptica a partir de un requerimiento que define al
\emph{SLA}.

Se ha evaluado cada uno de los dispositivos que conforma a la red, y
se ha obtenido el CAPEX y OPEX del primer año de una red moderna de
características comerciales.

Se concluye que el proyecto no solo es viable por la alta
escalabilidad que posee, sino que permite añadir capacidad de
transmisión de hasta 768 TB/s. Lograr una capacidad semejante con solo
fibra oscura requeriría instalar 79 fibras más con sus respectivos
equipos de transmisión y recepción, y no se podrían lograr establecer
garantías robustas a los clientes. Todos los estudios actuales indican
que el ancho de banda a nivel global está aumentando y las redes deben
tener capacidades para crecer sin aumentar significativamente los
costos de la red.
