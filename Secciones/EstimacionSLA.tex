\section{Estimación SLA}
\label{sec:SLA}

El acrónimo \emph{SLA} corresponde a las siglas de Service-level
agreement o de Acuerdo a Nivel de Servicio. Esto corresponde a un
contrato entre el proveedor y el cliente para asegurar que el servicio
se entregue correctamente ante un contexto legal. En el contrato se
especifican todos los puntos relevantes para garantizar una cierta
\emph{QoS} (calidad de servicio) para el cliente.

El \emph{SLA} es, en este caso, algo que se debe determinar de las
características del diseño del proyecto. Esto se debe a que existe una
cierta infraestructura que será aprovechada para cumplir los objetivos
buscados.

Entre los diversos elementos que permite garantizar el \emph{SLA} de
una red destacan:
\begin{itemize}
\item Servicios prestados
\item Parámetros de la QoS:
  \begin{itemize}
  \item Throughput
  \item Ancho de banda medio
  \item Latencia máxima
  \item Interrupciones máximas
  \end{itemize}
\item Tarifas y facturación
\end{itemize}

Una forma en que puede variar el \textbf{SLA} y que está considerado
entre las restricciones del proyecto es la probabilidad de cortes de
los cables.

\subsection{Efecto de la probabilidad de cortes de cable en el SLA}
\label{sec:cortes}

En una infraestructura de red oscura, un corte de cable entre dos data
centers significa que todas las conexiones entre esos data centers
quedan suspendidas hasta que alguien repare físicamente los cables
(introduciendo el valor del \emph{MTTR}). Ello repercute directamente
en el \emph{SLA}, reduciendo el índice de interrupciones máximas.

Afortunadamente, el desarrollo actual permite que acceder a la
tecnología de conmutación dinámica de red ópticas sea mucho más fácil
de lo que era hace unos años. Acceder a arquitecturas basadas en
\emph{WSON}, \emph{ROADM} y \emph{WSS} hoy no aumenta
significativamente el \emph{CAOEX}, \emph{OPEX} o la complejidad del
proyecto, pero sí afecta sustancialmente al \emph{SLA}, introduciendo
mejoras al índice de interrupciones máximas al reducir al mínimo el
\emph{MTBF}.

\subsection{Garantías en cuando al ancho de banda y \emph{throughput}
  de la red óptica}
\label{sec:anchodebanda}

Entre las especificaciones del proyecto se pide un cierto número y
tipo de servicios para comunicaciones entre los data centers. Los
equipos instalados tras el diseño propuesto en este informe tienen la
obligación de dejar instalada la capacidad pedida en cada caso.

Las capacidades pedidas por data center y por protocolo de
transmisión se muestran en la tabla.

\begin{table}[H]
  \centering
  \begin{tabular}{| l | c | c | c | c | c | c |}
    \hline
    \textbf{10GE} & \textbf{CNT} & \textbf{CDV} & \textbf{LONG} & \textbf{SMH} & \textbf{NNA} & \textbf{LCD} \\
    \hline
    \textbf{CNT}  & - & 20 & 10 & 4 & 4 & 6 \\
    \hline
    \textbf{CAD}  &   & - & 10 & 4 & 4 & 4 \\
    \hline
    \textbf{LONG} &   &   & - & 6 & 4 & 4 \\
    \hline
    \textbf{SMH}  &   &   &   & - & 4 & 4 \\
    \hline
    \textbf{NNA}  &   &   &   &   & - & 4 \\
    \hline
    \textbf{LCD}  &   &   &   &   &   & - \\
    \hline
  \end{tabular}
  \caption{Número de servicios entre data centers para protocolo 10GE}
  \label{tab:10ge}
\end{table}

% \begin{table}[H]
%   \centering
%   \begin{tabular}{| l | c | c | c | c | c | c |}
%     \hline
%     \textbf{10GE} & \textbf{CNT} & \textbf{CDV} & \textbf{LONG} & \textbf{SMH} & \textbf{NNA} & \textbf{LCD} \\
%     \hline
%     \textbf{CNT}  & - & 20 & 10 & 4 & 4 & 6 \\
%     \hline
%     \textbf{CAD}  &   & - & 10 & 4 & 4 & 4 \\
%     \hline
%     \textbf{LONG} &   &   & - & 6 & 4 & 4 \\
%     \hline
%     \textbf{SMH}  &   &   &   & - & 4 & 4 \\
%     \hline
%     \textbf{NNA}  &   &   &   &   & - & 4 \\
%     \hline
%     \textbf{LCD}  &   &   &   &   &   & - \\
%     \hline
%   \end{tabular}
%   \caption{Número de servicios entre data centers para protocolo 10GE}
%   \label{tab:}
% \end{table}
