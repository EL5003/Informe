\section{Caracterización de la red actual}
\label{sec:caracterizacion}

La red considerada para el comienzo de este proyecto consiste en un 
tendido de fibra óptica estándar tipo G.652.D (ver sección 
\ref{sec:marcoteorico}). Cada enlace de fibra óptica une un par de 
\emph{datacenters} de forma directa y sin estaciones intermedias mediante
cables de 96 fibras. Esta red no posee ningún tipo de modulador o 
elemento activo propio que la aproveche, sólo cuenta con los aparatos 
suministrados por terceros para su uso personal.

El tendido inicial es de 399 Km distribuídos en 9 tramos, de los
cuales tres son de distancias mayores a 40 Km. Este tendido tal y como 
está no tiene considerado ningún análisis que considere cortes de los 
cables, por lo que no existen garantías \emph{per se} que establezcan 
políticas claras de calidad de servicio. Por otro lado, es sabido que 
tramos de más de 40 Km. deben ser compensados en dispersión y atenuación. 
Por esa razón el tendido de fibra oscura no cumple tampoco con los SLA 
establecidos en calidad de la señal (ver sección \ref{sec:SLA}).

Por otro lado, no existe en la red inicial mecanismos íntegros de 
recepción, tales cómo equipos de \textit{Optical Distribution Frame (ODF)}, 
equipos Mux/Demux (\textit{WSS o ROADM}) con sus respectivos 
\textit{Racks Y Shelfs}, por lo que la red no resulta escalable en ningún 
sentido.

% Para la caracterización y planificación de la red, se consideran 3
% factores claves en todas sus etapas:

% \begin{itemize}
% \item Delay Introducido
% \item Distancia
% \item Costos Monetarios
% \end{itemize}

% Para la Fibra Oscura:

% Si el delay es considerado, se puede incluir cómo un costo adicional
% de distancia (sabiendo la relación entre cantidad de saltos y
% distancia), de tal forma que la minimización ocurra en torno a una
% sola variable.

% Una vez resueltos los costos en una sola variable (Saltos y
% Distancia), se procederá a minimizar la función costo usando el
% conocido algoritmo de Dijkstra.

% Adicionalmente a este procedimiento, para el caso de posibles cortes
% simultáneos, ha de hacerse un estudio probabilístico, dependiendo de
% que tan probable es que se corte una fibra según un umbral ``$\mu$'' se
% dispondrá de una o más rutas alternativas para esta.
