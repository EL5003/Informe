\section{Ingenier\'ia de Detalle}\label{sec:ingdetalle}

Es el conjunto de documentación técnica completa del proyecto que permite que la ejecución 
de éste sea entregada a un tercero y éste pueda efectuar la Instalación y Puesta en Marcha 
con mínimas variaciones.

La Instalación y Puesta en marcha de un proyecto no necesariamente debe ser ejecutada por 
el grupo de trabajo que prepara la Ingeniería de Detalle. El ejecutor puede ser otra sección 
dentro de la misma Empresa o una Empresa contratista (Outsourcing). Por este motivo la 
Ingeniería de detalle debe ser completa y muy precisa. No deben quedar detalles sin definir.

Dentro de la Ingeniería de detalle de un proyecto, y dependiendo de la naturaleza de éste, se 
pueden incluir los siguientes elementos:

Planos de planta de las estaciones, especificando ubicación y dimensiones de los racks 
que alojarán a los equipos nuevos Layout de los equipos, esto es, planos frontales de los racks indicando la disposición 
de los equipos.

Diagramas en bloque de los equipos indicando las interconexiones de los diferentes 
módulos (no se trata de los diagramas de circuitos electrónicos de cada tarjeta, información que los fabricantes no suelen entregar, sino que de las conexiones 
externas entre las diferentes tarjetas, para conseguir que los equipos trabajen en la 
forma deseada).

Diagrama de cross-conexiones entre las diferentes interfaces de línea y las puertas 
tributarias de los equipos. 
Plan de sincronización de la Red 
Planos de Planta Externa 
Memorias de cálculo 
 
 
El último elemento requiere de explicaciones adicionales. 
 
Una Memoria de Cálculo, como su nombre indica, es el resultado de los análisis y cálculos 
que efectúa un ingeniero especialista en una materia para determinar la forma correcta como 
debe ejecutarse una parte del proyecto para que el sistema funcione correctamente y de 
acuerdo a lo esperado. 
 
La memoria de cálculo puede ser de varios tipos según el proyecto. Algunos ejemplos: 
 
Memoria de cálculo de malla de tierra. Arroja como resultado especificaciones sobre la 
geometría de la malla, profundidad y número de las barras de cobre que deben 
enterrarse, valor de resistividad del terreno al cual debe llegarse. 
 
Memoria de cálculo de un radio enlace. Incluye un perfil del enlace, cálculos de niveles 
de señal, atenuaciones de las guías de onda y filtros, ganancia de las antenas, 
potencias de salida de los transmisores y nivel de recepción. Pero lo más importante es 
el cálculo de predicción de comportamiento del enlace en el peor mes del año, esto es 
\% del tiempo que el enlace estará indisponible (Tasa de error peor que 1E-03) y \% del 
tiempo que el enlace estará degradado (Tasa de error peor que 1E-06). Todos los 
métodos de cálculo y valores límites para los porcentajes mencionados están 
detallados por el ITU. 
 
Memoria de cálculo de enlaces por fibra óptica. Este tema es de importancia para el 
proyecto que les corresponde desarrollar. estos cálculos incluyen: balance de potencia; 
cálculo de razón señal a ruido óptica (OSNR) y cálculos de dispersión. 
 
Los resultados de estos cálculos determinan el diseño de la red y por lo tanto tiene 
impacto en el CAPEX y OPEX del proyecto: tipo de interfaces ópticas (estándar, de alta 
potencia, Ultra alta potencia), necesidad de usar Amplificadores Ópticos de Línea 
(OLA) o Regeneradores intermedios en un enlace. 
 
El impacto en el CAPEX es claro, si el diseño arroja que es necesario incluir una 
estación regeneradora en medio de un enlace entre estaciones ya existentes, el costo 
del proyecto se eleva notablemente. 
 
Por otra parte, un diseño muy audaz puede significar una reducción del margen 
disponible en el enlace, lo cual ante una degradación menor del cable de fibra puede 
hacer que el enlace se corte o degrade, obligando a efectuar mantenimiento con más 
frecuencia y elevando el OPEX. 
 
Memoria de cálculo de una torre. En un proyecto que incluya enlaces de microondas y 
emplazamiento de radio estaciones nuevas con sus receptivas torres, se debe incluir el 
cálculo de la estructura de la torre para asegurar su resistencia al peso de las antenas, 
al viento, etc. Naturalmente este cálculo cae dentro de las responsabilidades de un 
Ingeniero Civil no Eléctrico. 

