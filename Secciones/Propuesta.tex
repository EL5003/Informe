\section{Propuesta}
\label{sec:propuesta}

\subsection{Cotizaciones}
\label{sec:disenos}

En una primera instancia, varios proveedores de dispositivos ópticos
\emph{DWDM} fueron contactados para estudiar distintas propuestas con
equipos distintos. Contar con varias de estas cotizaciones habría
permitido comparar los costos de los equipos que irán instalados en
los nodos de la red y así habría permitido poder estimar de mejor
manera el costo de capital del proyecto.

Los proveedores de redes contactados fueron:
\begin{itemize}
\item Alcatel
\item Ericsson
\item Huawei
\item Nokia NSN
\item Siemens
\item Sonda (representantes de Cisco en Chile)
\end{itemize}

A la par de esta cotización, se desarrolló un modelo probabilístico
para la programación de los dispositivos \emph{ROADM} que asegura un
valor de mínimo de disponibilidad de la red; esto determina
principalmente el \emph{SLA} estimado de la red (mayores detalles se
exponen en la sección \ref{sec:SLA}). Este modelo probabilístico se
programará en la red con topología \emph{WSON} y controlada a través
de dispositivos \emph{WSS}, haciendo que el requerimiento en
\emph{SLA} sea en efecto una variable de entrada para el diseño.

% Para la definición de la red, fue necesario definir las rutas que
% conectan cada par de \textit{DataCenters}.  Para obtener un modelo
% robusto a cortes, es necesario que exista más de una ruta entre cada
% par de puntos o almenos cierta cantidad que garantize los
% requerimientos predefinidos.

% La forma en que se diseño esta red, utiliza ciertas características
% intrínsicas de la red de fibra oscura actual. Por ejemplo se ciñe a
% las tasas anuales de corte para la fibra utilizada (AEG-10
% \textbf{CORREGIR}), y también a las capacidades requeridas por cada
% protocolo GE y (balblalbal)


% El conjunto de rutas alternativas debe cumplir un criterio de umbral
% de seguridad, si no lo cumple se agregarán más rutas alternativas
% hasta que se logre dicho objetivo.

Hasta el día 23 de julio de 2014, no ha habido respuesta de los
proveedores de sistemas completos. Por ello y para tener un sistema
cotizado para la fecha de entrega, se cotizó utilizando solamente la
lista de precios de Ericsson, con la cual se contaba gracias a
contactos del docente del taller con dicha empresa.

\subsection{Propuesta final}
\label{sec:ppfinal}

Al final, la propuesta que se presenta en este informe consiste en una
red óptica \emph{DWDM} modular, compuesto por dispositivos, jumpers y
racks. Estos equipos se instalarán en los \emph{DC} tal y como se
detalla en la sección \ref{sec:ingdetalle}.

En total, se cotizaron los equipos y en la cantidad mostrada en la
tabla \ref{tab:propfinal}.

\begin{table}[H]
  \centering
  \begin{tabular}{| c | r |}
    \hline
    Equipo & Cantidad \\
    \hline
    WSS & 36 \\
    EDFA & 72 \\
    MUX & 6 \\
    DEMUX & 6 \\
    ODF & 6 \\
    Racks & 12 \\
    Transpondedores 10GE & 92 \\
    Transpondedores 4FC & 122 \\
    Fibra (jumpers) & 132 \\
    \hline
  \end{tabular}
  \caption{Cantidad de equipos cotizados.}
  \label{tab:propfinal}
\end{table}

\subsubsection{CAPEX}
\label{sec:capex}

El CAPEX del proyecto es 10.

\subsubsection{OPEX}
\label{sec:opex}

El OPEX del primer año de operación del proyecto funcionando se
componen de los siguientes puntos:
\begin{itemize}
\item Gasto energético
\item Gasto de mantenimiento
  \begin{itemize}
  \item Reparación de equipos
  \item Reparación de cableado
  \item Materiales de mantenimiento
  \end{itemize}
\item Gestión técnica (reconfiguraciones, evaluar agregar nuevos
  equipos, etc)
\end{itemize}

Los puntos anteriores se pueden estimar a precio de mercado. Esto es
lo que se hizo en la tabla \ref{tab:opex}.

\begin{table}[H]
  \centering
  \begin{tabular}{| c | c | c | c |}
    \hline{}
    Ítem & Precio unitario (por hora) & Unidades (horas) totales & Total USD \\
    \hline{}
    Gasto energético & 1.10 & 8760 & 9636 \\
    \hline{}
    Gasto mantenimiento equipos & 100 & 50 & 5000 \\
    \hline{}
    Gestión técnica & 50 & 48 & 2400 \\
    \hline{}
    Total & & & 17036 \\
    \hline
  \end{tabular}
  \caption{Detalle OPEX de un año tipo (válido para el primer año)}
  \label{tab:opex}
\end{table}