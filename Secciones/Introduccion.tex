
\section{Introducci\'on}
\label{sec:intro}

\subsection{Overview del proyecto}
\label{sec:overview}

Durante el semestre de otoño de 2014, en el marco del curso EL5004
(Taller de Proyecto), se encomendó a un grupo de estudiantes de
ingeniería eléctrica de la Universidad de Chile, la planificación,
diseño y evaluación de un proyecto para mejorar la capacidad de
comunicación entre data centers de la empresa Entel S.A. Para ello,
se aprovechó la infraestructura existente de fibra óptica oscura,
que une los data centers en cuestión y que actualmente utilizan los
clientes para acceder a sus planes contratados.

En la primera parte de este informe se discuten los puntos que hacen
razonable y necesario el cambio de paradigma de la red. Para ello, se
exponen las principales diferencias entre una red oscura y una activa,
demostrando las ventajas técnicas, económicas y financieras en favor
de esta última. También se justifica el proyecto exponiendo sus
ventajas cuantitativa y cualitativamente.

Más adelante, se lleva a cabo la caracterización de la red existente.
Este paso exige determinar las variables a manipular durante el 
desarrollo de la solución. A través de la caracterización, se 
evalúan los tipos de implementación tecnológica posibles según los 
criterios introducidos.

Finalmente se presenta el diseño propuesto, la ingeniería de detalle
del mismo y los costos (CAPEX y OPEX) que explican el proyecto. Se
redactaron conclusiones sobre el diseño y planteamientos futuros de
diseño para un mejor despliegue de la red.

En este informe se podrá encontrar el por qué el proyecto es viable y
conviene a Entel el llevarlo a cabo.

\subsection{Objetivos del proyecto}
\label{sec:objetivos}

El objetivo de este proyecto es estudiar tanto la factibilidad técnica
y económica como el diseño final de una red activa de fibra óptica
para la interconexión garantizada de data centers, aprovechando un
tendido de fibra oscura preexistente.

La factibilidad técnica introduce la necesidad de estudiar a fondo la
arquitectura y topología de una red de fibra óptica.

\subsection{Acrónimos}
\label{sec:acronimos}

En el informe se utilizarán las siguientes siglas recurrentemente:
\begin{itemize}
\setlength{\itemsep}{0pt}
\item \itab{\textbf{FO:}} \tab{Fibra Óptica}
\item \itab{\textbf{GE:}} \tab{Gigabit Ethernet}
\item \itab{\textbf{DC:}} \tab{\emph{Data center}} \\
\-\ \tab{Centro de Datos}
\item \itab{\textbf{DWDM:}} \tab{\emph{Dense Wavelength Division Multiplexing}} \\
\-\ \tab{Multiplexación Densa por División de Longitudes de Onda}
\item \itab{\textbf{MTBF:}} \tab{\emph{Mean Time Between Failures}} \\
\-\ \tab{Tiempo Medio Entre Fallas}
\item \itab{\textbf{MTTR:}} \tab{\emph{Mean Time To Repair}} \\
\-\ \tab{Tiempo Medio Para Reparación}
\item \itab{\textbf{ODF:}} \tab{\emph{Optical Distribution Frame}} \\
\-\ \tab{Organizador de Fibra Óptica}
\item \itab{\textbf{SLA:}} \tab{\emph{Service Level Agreement}} \\
\-\ \tab{Acuerdo de Nivel de Servicio}
\item \itab{\textbf{OADM:}} \tab{\emph{Optical Add/Drop Multiplexer}} \\
\-\ \tab{Multiplexador de Agregación/Remoción Óptica}
\item \itab{\textbf{ROADM:}} \tab{\emph{Reconfigurable Optical Add/Drop Multiplexer}} \\
\-\ \tab{Multiplexador Reconfigurable de Agregación/Remoción Óptica}
\item \itab{\textbf{WSON:}} \tab{\emph{Wavelength Switched Optical Network}} \\
\-\ \tab{Red Óptica Conmutada por Longitud de Onda}
\item \itab{\textbf{WSS:}} \tab{\emph{Wavelength Selective Switching}} \\
\-\ \tab{Conmutación Selectiva por Longitud de Onda}
\item \itab{\textbf{OLA:}} \tab{\emph{Optical Line Amplifier}} \\
\-\ \tab{Amplificador Óptico de Línea}
\item \itab{\textbf{OSC:}} \tab{\emph{Optical Service Channel}} \\
\-\ \tab{Canal Óptico de Servicio}
\end{itemize}