\section{Introducci�n}\label{sec:intro}

Durante el semestre de oto�o de 2014, en el marco del curso EL5004
Taller de Proyecto, se encomend� a un grupo de estudiantes de
ingenier�a el�ctrica de la Universidad de Chile, la planificaci�n,
dise�o y evaluaci�n de un proyecto para mejorar la capacidad de
comunicaci�n entre data centers de la empresa Entel S.A. Para ello,
se aprovech� la infraestructura existente de fibra �ptica oscura,
que une los data centers en cuesti�n y que actualmente utilizan los
clientes de forma directa para comunicaciones privadas, lo que implica
una capacidad limitada de oferta disponible.

En la primera parte de este informe se discutieron los puntos que
hacen razonable y necesario el cambio de paradigma de la red. Para
ello, se discutir�n las principales diferencias entre ambos tipos de
fibra: oscura y activa. Se marcaron los puntos que hacen poco
conveniente no realizar el proyecto, y a la vez se discutieron tanto
los aspectos t�cnicos como los econ�micos y financieros en favor de
las redes activas.

M�s adelante, se realiz� la caracterizaci�n de la red. Este paso
incluy� determinar las variables a manipular durante el desarrollo de
la soluci�n. A trav�s de la caracterizaci�n, se discutieron los tipos
de implementaci�n tecnol�gica posibles seg�n los criterios
introducidos.

Finalmente se presentaron las inversiones OPEX y CAPEX necesarias
para el proyecto adem�s de las conclusiones y planteamientos futuros
de dise�o para un mejor despliegue de la red.
