
\section{Introducci\'on}
\label{sec:intro}

\subsection{Overview del proyecto}
\label{sec:overview}

Durante el semestre de otoño de 2014, en el marco del curso EL5004
(Taller de Proyecto), se encomendó a un grupo de estudiantes de
ingeniería eléctrica de la Universidad de Chile, la planificación,
diseño y evaluación de un proyecto para mejorar la capacidad de
comunicación entre data centers de la empresa Entel S.A. Para ello,
se aprovechó la infraestructura existente de fibra óptica oscura,
que une los data centers en cuestión y que actualmente utilizan los
clientes de forma directa para acceder a sus planes contratados.

En la primera parte de este informe se discuten los puntos que
hacen razonable y necesario el cambio de paradigma de la red. Para
ello, se exponen las principales diferencias entre una red oscura
y una activa, demostrando las ventajas técnicas, económicas y
financieras en favor de esta última.

Más adelante, se lleva a cabo la caracterización de la red existente.
Este paso exige determinar las variables a manipular durante el 
desarrollo de la solución. A través de la caracterización, se 
evalúan los tipos de implementación tecnológica posibles según los 
criterios introducidos.

Finalmente se presentan las inversiones OPEX y CAPEX necesarias
para el proyecto además de las conclusiones y planteamientos futuros
de diseño para un mejor despliegue de la red.

\subsection{Objetivos del proyecto}
\label{sec:objetivos}

Estudiar tanto la factibilidad técnica y económica como el diseño
final de una red activa de fibra óptica para la interconexión
garantizada de data centers, aprovechando un tendido de fibra oscura
preexistente.

La factibilidad técnica introduce la necesidad de estudiar a fondo la
arquitectura de una red de fibra óptica.

\subsection{Acrónimos}
\label{sec:acronimos}

En el informe se utilizarán las siguientes siglas:
\begin{itemize}
\item \textbf{DC} \emph{Data center} - Centro de datos
\item \textbf{DWDM} \emph{Dense Wavelength Division Multiplexing} - Multiplexación por división de longitudes de onda densas
\item \textbf{FO} Fibra óptica
\item \textbf{GE} Gigabit Ethernet
\item \textbf{MTBF} \emph{Mean Time Between Failures} - Tiempo medio entre fallas
\item \textbf{MTTR} \emph{Mean Time To Repair} - Tiempo medio para reparación
\item \textbf{OADM} \emph{Optical add/drop multiplexor} - Multiplexador óptico de agregación/remoción
\item \textbf{ODF} \emph{Optical Distribution Frame} - Organizador de fibra óptica
\item \textbf{SLA} \emph{Service Level Agreement} - Acuerdo de nivel de servicio
\item \textbf{ROADM} \emph{Reconfigurable optical add/drop multiplexor} - Multiplexador óptico de agregación/remoción
\item \textbf{WSON} \emph{Wavelength Switched Optical Network} - Red óptica conmutada por longitud de onda
\item \textbf{WSS} \emph{Wavelength Selective Switching} - Conmutación de longitud de onda selectiva
\end{itemize}