\section{Introducci\'on}\label{sec:intro}

Durante el semestre de otoño de 2014, en el marco del curso EL5004 Taller de Proyecto, se encomendó a un grupo de estudiantes de ingeniería eléctrica de la Universidad de Chile, la planificación, diseño y evaluación de un proyecto para mejorar la capacidad de comunicación entre data centers de la empresa Entel S.A.. Para ello, se aprovechó la infraestructura existente de fibra óptica oscura, que une los data centers en cuestión y que actualmente utilizan los clientes de forma directa para comunicaciones privadas, lo que implica una capacidad limitada de oferta disponible.

En la primera parte de este informe se discutirán los puntos que hacen razonable y necesario el cambio de paradigma de la red. Para ello, se discutirán las principales diferencias entre ambos tipos de fibra: oscura y activa. Se marcarán los puntos que hacen poco conveniente no realizar el proyecto, y a la vez se discutirán tanto los aspectos técnicos como los económicos y financieros en favor de las redes activas.

Más adelante, se realizará la caracterización de la red. Este paso incluye determinar las variables a manipular durante el desarrollo de la solución. A través de la caracterización, se discutirán los tipos de implementación tecnológica posibles según los criterios introducidos.

Finalmente se presentarán las inversiones OPEX y CAPEX necesarias para el proyecto además de las conclusiones y planteamientos futuros de diseño para un mejor despliegue de la red.
