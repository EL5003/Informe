\section{Propuesta}
\label{sec:propuesta}



\subsection{Diseños propuestos}
\label{sec:disenos}

La propuesta consiste en desarrollar una red dinámica y eficiente que
sea robusta ante posibles fallas en la interconexión física de los
\emph{datacenters}. Para esto fueron contactados varios proveedores de
dispositivos ópticos \emph{DWDM} con el fin de cotizar los equipos que 
irán instalados en los nodos de la red y así poder contar con un valor 
estimado de el costo de capital del proyecto.

Los proveedores de redes contactados fueron:
\begin{itemize}
\item Alcatel
\item Ericsson
\item Huawei
\item Nokia NSN
\item Siemens
\item Sonda (representantes de Cisco en Chile)
\end{itemize}

A la par de esta cotización, se desarrolló un modelo probabilístico 
para la programación de los dispositivos \emph{ROADM}
que asegura un valor de mínimo de disponibilidad de la red; esto
determina principalmente el \emph{SLA} estimado de la red (mayores
detalles se exponen en la sección \ref{sec:SLA}).

Este modelo probabilístico se programará en el controlador
\emph{ROADM} de la red, haciendo que el requerimiento en \emph{SLA} de
la red sea en efecto una variable de entrada para el diseño.

% Para la definición de la red, fue necesario definir las rutas que
% conectan cada par de \textit{DataCenters}.  Para obtener un modelo
% robusto a cortes, es necesario que exista más de una ruta entre cada
% par de puntos o almenos cierta cantidad que garantize los
% requerimientos predefinidos.

% La forma en que se diseño esta red, utiliza ciertas características
% intrínsicas de la red de fibra oscura actual. Por ejemplo se ciñe a
% las tasas anuales de corte para la fibra utilizada (AEG-10
% \textbf{CORREGIR}), y también a las capacidades requeridas por cada
% protocolo GE y (balblalbal)


% El conjunto de rutas alternativas debe cumplir un criterio de umbral
% de seguridad, si no lo cumple se agregarán más rutas alternativas
% hasta que se logre dicho objetivo.

Hasta el día 18 de julio e 2014, no ha habido respuesta de los
proveedores de sistemas completos. Por ello y para tener un sistema
cotizado para la fecha de entrega, se cotizó en línea en tiendas
proveedoras de equipos separados.

\subsection{Propuesta final}
\label{sec:ppfinal}

Al final, la propuesta que se presenta en este informe consiste en una
red óptica \emph{DWDM} modular, compuesto por dispositivos y
racks. Estos equipos se instalarán en los \emph{DC}.

\subsubsection{CAPEX}
\label{sec:capex}

El CAPEX del proyecto es 10.

\subsubsection{OPEX}
\label{sec:opex}

El OPEX del primer año de operación del proyecto funcionando se
componen de los siguientes puntos:
\begin{itemize}
\item Gasto energético
\item Gasto de mantenimiento
  \begin{itemize}
  \item Reparación de equipos
  \item Reparación de cableado
  \item Materiales de mantenimiento
  \end{itemize}
\item Gestión técnica (reconfiguraciones, evaluar agregar nuevos
  equipos, etc)
\end{itemize}

Los puntos anteriores se pueden estimar a precio de mercado. Esto es
lo que se hizo en la tabla \ref{tab:opex}.

\begin{table}[H]
  \centering
  \begin{tabular}{| c | c | c | c |}
    \hline{}
    Ítem & Precio unitario (por hora) & Unidades (horas) totales & Total USD \\
    \hline{}
    Gasto energético & 1.10 & 8760 & 9636 \\
    \hline{}
    Gasto mantenimiento equipos & 100 & 50 & 5000 \\
    \hline{}
    Gestión técnica & 50 & 48 & 2400 \\
    \hline{}
    Total & & & 17036 \\
    \hline
  \end{tabular}
  \caption{Detalle OPEX de un año tipo (válido para el primer año)}
  \label{tab:opex}
\end{table}