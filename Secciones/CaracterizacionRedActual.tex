\section{Caracterización de la red actual}
\label{sec:caracterizacion}

Para la caracterización y planificación de la red, se consideran 3
factores claves en todas sus etapas:

\begin{itemize}
\item Delay Introducido
\item Distancia
\item Costos Monetarios
\end{itemize}

Para la Fibra Oscura:

Si el delay es considerado, se puede incluir cómo un costo adicional
de distancia (sabiendo la relación entre cantidad de saltos y
distancia), de tal forma que la minimización ocurra en torno a una
sola variable.

Una vez resueltos los costos en una sola variable (Saltos y
Distancia), se procederá a minimizar la función costo usando el
conocido algoritmo de Dijkstra.

Adicionalmente a este procedimiento, para el caso de posibles cortes
simultáneos, ha de hacerse un estudio probabilístico, dependiendo de
que tan probable es que se corte una fibra según un umbral ``$\mu$'' se
dispondrá de una o más rutas alternativas para esta.
