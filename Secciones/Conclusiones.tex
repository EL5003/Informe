\section{Conclusiones}\label{sec:conclusiones}

Las redes ópticas pueden ser diseñadas de manera flexible usando
tecnologías que permiten multiplexar y manejar una gran cantidad de
señales de forma confiable y eficiente en una sola fibra óptica. Esto
se puede hacer de forma escalonada, instalando equipos que conformen
una red óptica \emph{DWDM}.

Los costos de estas implementaciones no son nada fuera de lo común y
día a día disminuyen más los precios. Los sistemas de fibra oscura han
perdido terreno en el mercado por su baja escalabilidad y poca
flexibilidad.

Se han agregado funcionalidades al tendido inicial de fibra oscura que
se traducen en una mejor calidad de servicio, al idear una forma de
diseñar la red óptica a partir de un requerimiento que define al
\emph{SLA}.

Se ha evaluado cada uno de los dispositivos que conforma a la red de
manera descentralizada, y se ha obtenido el CAPEX y OPEX del primer
año de una red de estas características.
