\section{Propuesta}\label{sec:propuesta}

La propuesta consiste en desarrollar una red dinamica eficiente que sea robusta ante posibles fallas en la interconexión física de los datacenters. Para esto se ha desarrollado un modelo basado probabilístico que asegura el total funcionamiento de la red para un cierto umbral de seguridad.

\begin{figure}[h]
\centering
\includegraphics[width=0.7\textwidth]{Imagenes/datacenters.pdf}
\caption{Diagrama d}
\end{figure}

Para la definición de la red, fue necesario definir las rutas que conectan cada par de \textit{DataCenters}.
Para obtener un modelo robusto a cortes, es necesario que exista más de una ruta entre cada par de puntos o almenos cierta cantidad que garantize los requerimientos predefinidos. 

La forma en que se diseño esta red, utiliza ciertas características intrínsicas de la red de fibra oscura actual. Por ejemplo se ciñe a las tasas anuales de corte para la fibra utilizada (AEG-10 \textb{CORREGIR}), y también a las capacidades requeridas por cada protocolo GE y (balblalbal) 


El conjunto de rutas alternativas debe cumplir un criterio de umbral de seguridad, si no lo cumple se agregarán más rutas alternativas hasta que se logre dicho objetivo. 