\section{Introducción}
\label{sec:intro}

Durante el semestre de otoño de 2014, en el marco del curso EL5004
Taller de Proyecto, se encomendó a un grupo de estudiantes de
ingeniería eléctrica de la Universidad de Chile, la planificación,
diseño y evaluación de un proyecto para mejorar la capacidad de
comunicación entre data centers de la empresa Entel S.A. Para ello,
se aprovechó la infraestructura existente de fibra óptica oscura,
que une los data centers en cuestión y que actualmente utilizan los
clientes de forma directa para comunicaciones privadas, lo que implica
una capacidad limitada de oferta disponible.

En la primera parte de este informe se discutieron los puntos que
hacen razonable y necesario el cambio de paradigma de la red. Para
ello, se discutirán las principales diferencias entre ambos tipos de
fibra: oscura y activa. Se marcaron los puntos que hacen poco
conveniente no realizar el proyecto, y a la vez se discutieron tanto
los aspectos técnicos como los económicos y financieros en favor de
las redes activas.

Más adelante, se realizó la caracterización de la red. Este paso
incluyó determinar las variables a manipular durante el desarrollo de
la solución. A través de la caracterización, se discutieron los tipos
de implementación tecnológica posibles según los criterios
introducidos.

Finalmente se presentaron las inversiones OPEX y CAPEX necesarias
para el proyecto además de las conclusiones y planteamientos futuros
de diseño para un mejor despliegue de la red.
